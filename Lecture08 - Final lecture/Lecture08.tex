\documentclass{beamer} %[handout]
\usetheme{CambridgeUS}
\usecolortheme{whale}
\usepackage{color}
\usepackage{xcolor}
\usepackage[makeroom]{cancel}
\usepackage{graphicx} 
\setbeamercolor{frametitle}{fg=black}
\usepackage{mathtools}
\usepackage{amssymb}
\usepackage{amsmath}
\usepackage{amsthm}
\usepackage{graphicx}
\usepackage{fancyvrb}
\usepackage{listings}
\usepackage{tikz}
\usepackage[utf8]{inputenc}

\usetikzlibrary{shapes}

\usetikzlibrary{decorations.text}

\lstset{frame=tb,
  language=R,
  aboveskip=3mm,
  belowskip=3mm,
  showstringspaces=false,
  columns=flexible,
  basicstyle={\tiny\ttfamily},
  numbers=none,
  numberstyle=\tiny\color{gray},
%  keywordstyle=\color{blue},
%  identifierstyle=\color{yellow},
  breaklines=true,
    literate={->}{$\rightarrow$}{2}
           {°}{$\epsilon$}{1},
  breakatwhitespace=true
  tabsize=3
}

\usetikzlibrary{arrows,positioning} 
\tikzset{
    %Define standard arrow tip
    >=stealth',
    %Define style for boxes
    punkt/.style={
           rectangle,
           rounded corners,
           draw=black, very thick,
           text width=6.5em,
           minimum height=2em,
           text centered},
    % Define arrow style
    pil/.style={
           ->,
           thick,
           shorten <=2pt,
           shorten >=2pt,}
}

\makeatletter
\def\th@mystyle{%
    \normalfont % body font
    \setbeamercolor{block title example}{bg=orange,fg=white}
    \setbeamercolor{block body example}{bg=orange!20,fg=black}
    \def\inserttheoremblockenv{exampleblock}
  }
\makeatother
\theoremstyle{mystyle}
\newtheorem*{remark}{Example}

\makeatletter
\def\th@mystylet{%
    \normalfont % body font
    \setbeamercolor{block title example}{bg=purple,fg=white}
    \setbeamercolor{block body example}{bg=purple!20,fg=black}
    \def\inserttheoremblockenv{exampleblock}
  }
\makeatother
\theoremstyle{mystylet}
\newtheorem*{analysis}{Example}

\date{}
\author[Bayesian statistics]{Erik \v{S}trumbelj\\2019}

\linespread{1.2}
\renewcommand{\thefootnote}{\roman{footnote}}
%\setbeameroption{show notes}
% set new colors according to AMU design
% http://dircom.univ-amu.fr/sites/dircom.univ-amu.fr/files/charte_graphique_generale_amu_0.pdf
\definecolor{AMUDarkBlue}{RGB}{0,103,186}
\definecolor{AMULightBlue}{RGB}{110,188,253}
\definecolor{AMUYellow}{RGB}{246,189,23}
%\usecolortheme[named=AMUDarkBlue]{structure}
% top bar with section navigation
%\setbeamercolor{palette tertiary}{bg=black,fg=white}
% 2nd top bar with subsection navigation (only if subsection=true)
%\setbeamercolor{palette secondary}{bg=AMUDarkBlue!50,fg=white}
% title bar of each slide
%\setbeamercolor{titlelike}{bg=AMUDarkBlue,fg=white}
%\setbeamercolor{title}{bg=AMUDarkBlue,fg=white}
%\setbeamercolor{subtitle}{fg=white}
\setbeamercolor{block title}{bg=AMUDarkBlue,fg=white}
\setbeamercolor{block body}{bg=AMUDarkBlue!30}
\setbeamercolor{block title example}{bg=AMUYellow,fg=white}
\setbeamercolor{block body example}{bg=AMUYellow!30}
\setbeamercolor{block title alerted}{bg=red,fg=white}
\setbeamercolor{block body alerted}{bg=red!30}
% \setbeamercolor{colorbox}{bg=HUSand!30}
% http://tex.stackexchange.com/questions/74303
\usepackage{etoolbox}
\AtBeginEnvironment{exampleblock}{%
  \setbeamercolor{itemize item}{fg=AMUYellow!70}
}
\AtBeginEnvironment{alertblock}{%
  \setbeamercolor{itemize item}{fg=red!70}
}
\AtBeginEnvironment{block}{%
  \setbeamercolor{itemize item}{fg=AMUDarkBlue!70}
}
% \rowcolors{2}{Gray!30}{white}

% apply default template partials
\useinnertheme{rectangles}
\useoutertheme[subsection=false]{miniframes}
\setbeamertemplate{navigation symbols}{}
% reset the footer: show slide numbers
\setbeamertemplate{footline}{%
  \begin{beamercolorbox}[colsep=1.5pt]{upper separation line foot}
  \end{beamercolorbox}
  \begin{beamercolorbox}[ht=2.5ex,dp=1.125ex,%
    leftskip=.3cm,rightskip=.3cm plus1fil]{title in head/foot}%
    {\usebeamerfont{title in head/foot}\insertshorttitle}%
    \,
    {\usebeamerfont{author in head/foot}(\insertshortauthor)}%
    \hfill
    {\usebeamerfont{framenumber in head/foot}\textbf{\insertframenumber}}%
  \end{beamercolorbox}%
  \begin{beamercolorbox}[colsep=1.5pt]{lower separation line foot}
  \end{beamercolorbox}
}
% emphasizing means text in default accent color
\renewcommand{\emph}[1]{\textcolor{structure}{#1}}

% \usepackage{microtype} % enable this for improved text typesetting
\usepackage{hyperref} % this is of \href{}{} or \url{}
\usepackage{booktabs} % this is for \midrule \toprule etc in tables


\usetikzlibrary{bayesnet}

\title[Final lecture]{Final lecture}


\begin{document}

\begin{frame}
Bayesian statistics
\titlepage
\end{frame}




\begin{frame}{Further reading}

\begin{scriptsize}

\begin{itemize}

\item Andrew Gelman, Jennifer Hill, and Masanao Yajima. Why we (usually) don't have to worry about multiple comparisons. Journal of Research on Educational Effectiveness, 5(2), 189-211, 2012.

\item Andrew Gelman, Jessica Hwang, and Aki Vehtari. Understanding predictive
information criteria for Bayesian models. Statistics and Computing, 24(6):
997–1016, 2014.

\item Aki Vehtari, Andrew Gelman, and Jonah Gabry. Practical Bayesian model
evaluation using leave-one-out cross-validation and WAIC. Statistics and
Computing, 27(5):1413–1432, 2017.

\item John K Kruschke. Bayesian estimation supersedes the t test. Journal of Experimental
Psychology: General, 142(2):573, 2013.

\end{itemize}
\end{scriptsize}
\end{frame}

\begin{frame}{Topics we've covered}

\begin{itemize}
\item Bayesian view of probability.
\item Bayesian learning: Bayes' theorem, prior, likelihood, posterior.
\item Computation: conjugacy, MC, MCMC (semi-conjugacy, Gibbs).
\item MCMC in practice: autocovariance, approximation error, traceplot, effective sample size.
\item Fitting, overfitting.
\item Some standard models: binomial test for proportions, t-test \& ANOVA (equivalent), lm, glm, mixed-effects modelling...
\item Software: Stan, \texttt{rstanarm}, \texttt{mcmcse}...
\end{itemize}

\end{frame}


\begin{frame}{Where do we go from here}

\begin{small}
  \begin{block}{Applying Bayesian statistics as a research tool in your field}
    \begin{itemize}
    \item Kruschke: Doing Bayesian data analysis.
    \item McElreath: Statistical Rethinking: A Bayesian Course with Examples in R and Stan.
    \item Learn R, rstanarm, Stan.
    \item (Gelman...: Bayesian data analysis.)
    \end{itemize}
  \end{block}
  
  \begin{alertblock}{Specializing in Bayesian statistics}
    \begin{itemize}
      \item Ross: First course in probability theory.
      \item Kadane: Principles of uncertainty (subjective Bayesian view).
      \item Robert: The Bayesian choice (objective Bayesian view).
    \end{itemize}

  \end{alertblock}

\end{small}

\end{frame}


\begin{frame}{Topics we've covered}

\begin{itemize}
\item Bayesian view of probability.
\item Bayesian learning: Bayes' theorem, prior, likelihood, posterior.
\item Computation: conjugacy, MC, MCMC (semi-conjugacy, Gibbs).
\item MCMC in practice: autocovariance, approximation error, traceplot, effective sample size.
\item Fitting, overfitting.
\item Some standard models: binomial test for proportions, t-test \& ANOVA (equivalent), lm, glm, mixed-effects modelling...
\item Software: Stan, \texttt{rstanarm}, \texttt{mcmcse}...
\end{itemize}

\end{frame}


\end{document}
